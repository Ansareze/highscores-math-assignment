\documentclass[12pt]{article}
\usepackage{amsmath}
\usepackage{amsfonts}
\usepackage{amssymb}
\usepackage{geometry}
\usepackage{booktabs}
\usepackage{graphicx}
\usepackage{enumitem}

\geometry{margin=1in}

\title{Math Assessment Questions}
\author{Generated by AI}
\date{\today}

\begin{document}

\maketitle

\section*{Question 1: Ice Cream Sundae Combination Menu}

\subsection*{Description}
Students must calculate the total number of possible ice cream sundae combinations using a table of ice cream flavors and topping options.

\subsection*{Question}
An ice cream parlor is creating their sundae menu. They offer 5 different ice cream flavors and 8 different toppings. The table below shows the available options:

\begin{center}
\begin{tabular}{ll}
\toprule
\textbf{Ice Cream Flavors} & \textbf{Topping Options} \\
\midrule
Vanilla & Sprinkles \\
Chocolate & Nuts \\
Strawberry & Whipped Cream \\
Mint Chocolate Chip & Chocolate Chips \\
Cookies \& Cream & Caramel Sauce \\
& Cherry \\
& Marshmallows \\
& Gummy Bears \\
\bottomrule
\end{tabular}
\end{center}

How many different sundae combinations can be made using one ice cream flavor and one topping?

\subsection*{Instructions}
Calculate the total number of possible sundae combinations by multiplying the number of ice cream flavors by the number of topping options.

\subsection*{Difficulty}
Moderate

\subsection*{Multiple Choice Options}
\begin{enumerate}[label=(\Alph*)]
\item 13
\item 20
\item \textbf{40} \quad (Correct Answer)
\item 45
\end{enumerate}

\subsection*{Explanation}
To find the total number of sundae combinations, we use the multiplication principle. Since each ice cream flavor can be paired with each topping option, we multiply the number of flavors by the number of toppings:

$$5 \times 8 = 40$$

This means there are 40 different possible sundae combinations.

\subsection*{Curriculum Information}
\begin{itemize}
\item \textbf{Subject}: Quantitative Math
\item \textbf{Unit}: Problem Solving
\item \textbf{Topic}: Counting \& Arrangement Problems
\item \textbf{Points}: 1
\end{itemize}

\newpage

\section*{Question 2: Rectangular Box Packaging for Tennis Balls}

\subsection*{Description}
Students must calculate the minimum dimensions of a rectangular box needed to hold a specific number of tennis balls arranged in a grid pattern.

\subsection*{Question}
A sports equipment company needs to package 15 tennis balls in a rectangular box. Each tennis ball has a diameter of $6.7 \text{ cm}$. The balls will be arranged in a $3 \times 5$ grid pattern (3 rows, 5 columns) with no gaps between them.

\begin{center}
\textit{[Image: diagram of 15 tennis balls arranged in a 3×4 grid inside a rectangular box, showing the top view with balls touching each other]}
\end{center}

What are the minimum dimensions (length $\times$ width $\times$ height) of the rectangular box needed to hold all tennis balls?

\subsection*{Instructions}
Calculate the minimum box dimensions by considering the diameter of the tennis balls for length and width, and the diameter for the box height since balls are spherical.

\subsection*{Difficulty}
Moderate

\subsection*{Multiple Choice Options}
\begin{enumerate}[label=(\Alph*)]
\item $20.1 \text{ cm} \times 13.4 \text{ cm} \times 6.7 \text{ cm}$
\item $26.8 \text{ cm} \times 20.1 \text{ cm} \times 6.7 \text{ cm}$
\item \textbf{$33.5 \text{ cm} \times 20.1 \text{ cm} \times 6.7 \text{ cm}$} \quad (Correct Answer)
\item $40.2 \text{ cm} \times 26.8 \text{ cm} \times 13.4 \text{ cm}$
\end{enumerate}

\subsection*{Explanation}
To find the minimum box dimensions, we need to calculate:

\begin{itemize}
\item \textbf{Length}: $5 \text{ balls} \times \text{diameter} = 5 \times 6.7 = 33.5 \text{ cm}$
\item \textbf{Width}: $3 \text{ balls} \times \text{diameter} = 3 \times 6.7 = 20.1 \text{ cm}$
\item \textbf{Height}: $6.7 \text{ cm}$ (same as ball diameter)
\end{itemize}

Therefore, the minimum box dimensions are:
$$33.5 \text{ cm} \times 20.1 \text{ cm} \times 6.7 \text{ cm}$$

\subsection*{Curriculum Information}
\begin{itemize}
\item \textbf{Subject}: Quantitative Math
\item \textbf{Unit}: Geometry and Measurement
\item \textbf{Topic}: Area \& Volume
\item \textbf{Points}: 1
\end{itemize}

\end{document}
